\documentclass[]{article}
\usepackage{lmodern}
\usepackage{amssymb,amsmath}
\usepackage{ifxetex,ifluatex}
\usepackage{fixltx2e} % provides \textsubscript
\ifnum 0\ifxetex 1\fi\ifluatex 1\fi=0 % if pdftex
  \usepackage[T1]{fontenc}
  \usepackage[utf8]{inputenc}
\else % if luatex or xelatex
  \ifxetex
    \usepackage{mathspec}
  \else
    \usepackage{fontspec}
  \fi
  \defaultfontfeatures{Ligatures=TeX,Scale=MatchLowercase}
\fi
% use upquote if available, for straight quotes in verbatim environments
\IfFileExists{upquote.sty}{\usepackage{upquote}}{}
% use microtype if available
\IfFileExists{microtype.sty}{%
\usepackage{microtype}
\UseMicrotypeSet[protrusion]{basicmath} % disable protrusion for tt fonts
}{}
\usepackage[margin=1in]{geometry}
\usepackage{hyperref}
\hypersetup{unicode=true,
            pdftitle={COMP2550/COMP4450/COMP6445 - Quantitative Methods Lab Tutorial},
            pdfauthor={Dr Dongwoo Kim (ANU)},
            pdfborder={0 0 0},
            breaklinks=true}
\urlstyle{same}  % don't use monospace font for urls
\usepackage{color}
\usepackage{fancyvrb}
\newcommand{\VerbBar}{|}
\newcommand{\VERB}{\Verb[commandchars=\\\{\}]}
\DefineVerbatimEnvironment{Highlighting}{Verbatim}{commandchars=\\\{\}}
% Add ',fontsize=\small' for more characters per line
\usepackage{framed}
\definecolor{shadecolor}{RGB}{248,248,248}
\newenvironment{Shaded}{\begin{snugshade}}{\end{snugshade}}
\newcommand{\AlertTok}[1]{\textcolor[rgb]{0.94,0.16,0.16}{#1}}
\newcommand{\AnnotationTok}[1]{\textcolor[rgb]{0.56,0.35,0.01}{\textbf{\textit{#1}}}}
\newcommand{\AttributeTok}[1]{\textcolor[rgb]{0.77,0.63,0.00}{#1}}
\newcommand{\BaseNTok}[1]{\textcolor[rgb]{0.00,0.00,0.81}{#1}}
\newcommand{\BuiltInTok}[1]{#1}
\newcommand{\CharTok}[1]{\textcolor[rgb]{0.31,0.60,0.02}{#1}}
\newcommand{\CommentTok}[1]{\textcolor[rgb]{0.56,0.35,0.01}{\textit{#1}}}
\newcommand{\CommentVarTok}[1]{\textcolor[rgb]{0.56,0.35,0.01}{\textbf{\textit{#1}}}}
\newcommand{\ConstantTok}[1]{\textcolor[rgb]{0.00,0.00,0.00}{#1}}
\newcommand{\ControlFlowTok}[1]{\textcolor[rgb]{0.13,0.29,0.53}{\textbf{#1}}}
\newcommand{\DataTypeTok}[1]{\textcolor[rgb]{0.13,0.29,0.53}{#1}}
\newcommand{\DecValTok}[1]{\textcolor[rgb]{0.00,0.00,0.81}{#1}}
\newcommand{\DocumentationTok}[1]{\textcolor[rgb]{0.56,0.35,0.01}{\textbf{\textit{#1}}}}
\newcommand{\ErrorTok}[1]{\textcolor[rgb]{0.64,0.00,0.00}{\textbf{#1}}}
\newcommand{\ExtensionTok}[1]{#1}
\newcommand{\FloatTok}[1]{\textcolor[rgb]{0.00,0.00,0.81}{#1}}
\newcommand{\FunctionTok}[1]{\textcolor[rgb]{0.00,0.00,0.00}{#1}}
\newcommand{\ImportTok}[1]{#1}
\newcommand{\InformationTok}[1]{\textcolor[rgb]{0.56,0.35,0.01}{\textbf{\textit{#1}}}}
\newcommand{\KeywordTok}[1]{\textcolor[rgb]{0.13,0.29,0.53}{\textbf{#1}}}
\newcommand{\NormalTok}[1]{#1}
\newcommand{\OperatorTok}[1]{\textcolor[rgb]{0.81,0.36,0.00}{\textbf{#1}}}
\newcommand{\OtherTok}[1]{\textcolor[rgb]{0.56,0.35,0.01}{#1}}
\newcommand{\PreprocessorTok}[1]{\textcolor[rgb]{0.56,0.35,0.01}{\textit{#1}}}
\newcommand{\RegionMarkerTok}[1]{#1}
\newcommand{\SpecialCharTok}[1]{\textcolor[rgb]{0.00,0.00,0.00}{#1}}
\newcommand{\SpecialStringTok}[1]{\textcolor[rgb]{0.31,0.60,0.02}{#1}}
\newcommand{\StringTok}[1]{\textcolor[rgb]{0.31,0.60,0.02}{#1}}
\newcommand{\VariableTok}[1]{\textcolor[rgb]{0.00,0.00,0.00}{#1}}
\newcommand{\VerbatimStringTok}[1]{\textcolor[rgb]{0.31,0.60,0.02}{#1}}
\newcommand{\WarningTok}[1]{\textcolor[rgb]{0.56,0.35,0.01}{\textbf{\textit{#1}}}}
\usepackage{graphicx,grffile}
\makeatletter
\def\maxwidth{\ifdim\Gin@nat@width>\linewidth\linewidth\else\Gin@nat@width\fi}
\def\maxheight{\ifdim\Gin@nat@height>\textheight\textheight\else\Gin@nat@height\fi}
\makeatother
% Scale images if necessary, so that they will not overflow the page
% margins by default, and it is still possible to overwrite the defaults
% using explicit options in \includegraphics[width, height, ...]{}
\setkeys{Gin}{width=\maxwidth,height=\maxheight,keepaspectratio}
\IfFileExists{parskip.sty}{%
\usepackage{parskip}
}{% else
\setlength{\parindent}{0pt}
\setlength{\parskip}{6pt plus 2pt minus 1pt}
}
\setlength{\emergencystretch}{3em}  % prevent overfull lines
\providecommand{\tightlist}{%
  \setlength{\itemsep}{0pt}\setlength{\parskip}{0pt}}
\setcounter{secnumdepth}{0}
% Redefines (sub)paragraphs to behave more like sections
\ifx\paragraph\undefined\else
\let\oldparagraph\paragraph
\renewcommand{\paragraph}[1]{\oldparagraph{#1}\mbox{}}
\fi
\ifx\subparagraph\undefined\else
\let\oldsubparagraph\subparagraph
\renewcommand{\subparagraph}[1]{\oldsubparagraph{#1}\mbox{}}
\fi

%%% Use protect on footnotes to avoid problems with footnotes in titles
\let\rmarkdownfootnote\footnote%
\def\footnote{\protect\rmarkdownfootnote}

%%% Change title format to be more compact
\usepackage{titling}

% Create subtitle command for use in maketitle
\newcommand{\subtitle}[1]{
  \posttitle{
    \begin{center}\large#1\end{center}
    }
}

\setlength{\droptitle}{-2em}

  \title{COMP2550/COMP4450/COMP6445 - Quantitative Methods Lab Tutorial}
    \pretitle{\vspace{\droptitle}\centering\huge}
  \posttitle{\par}
    \author{Dr Dongwoo Kim (ANU)}
    \preauthor{\centering\large\emph}
  \postauthor{\par}
      \predate{\centering\large\emph}
  \postdate{\par}
    \date{18 March 2019}


\begin{document}
\maketitle

\hypertarget{introduction}{%
\section{Introduction}\label{introduction}}

In this tutorial we will be learning how to do quantitative research in
a data science setting. We will be doing a small-scale project analysing
Twitter data about social bots (automated accounts posing as humans)
influencing public debate during the 1st U.S. Presidential Debate in
2016.

The dataset is taken from \href{https://arxiv.org/abs/1802.09808}{Rizoiu
et al (2018)}

`\#DebateNight: The Role and Influence of Socialbots on Twitter During
the 1st U.S. Presidential Debate' (retrieved from
\url{https://arxiv.org/abs/1802.09808})

The \emph{assignment} in Week 4 follows on from this data set, but moves
into the area of machine learning.

Learning outcomes:

\begin{enumerate}
\def\labelenumi{\arabic{enumi}.}
\tightlist
\item
  Import and wrangle quantitative data in the R language (a statistical
  programming language)
\item
  Basic knowledge of R programming for data science problems
\item
  Undertake descriptive statistical analysis of different data types
\item
  Plot data using graphs and label these appropriately
\item
  Undertake basic inferential statistics and evaluate statistical
  significance
\end{enumerate}

The dataset for this tutorial consists of a \emph{random sample of
100,000} observations derived from the `\#DEBATENIGHT' dataset in
\href{https://arxiv.org/abs/1802.09808}{Rizoiu et al (2018)}.

\hypertarget{what-is-the-debatenight-dataset}{%
\subsection{What is the \#DEBATENIGHT
dataset}\label{what-is-the-debatenight-dataset}}

The \#DEBATENIGHT dataset contains Twitter discussions that occurred
during the first 2016 U.S presidential debate between Hillary Clinton
and Donald Trump. Using the Twitter Firehose API, we collected all the
tweets (including retweets) that were authored during the two hour
period from 8.45pm to 10.45pm EDT, on 26 September 2016, and which
contain at least one of the hashtags: \#DebateNight, \#Debates2016,
\#election2016, \#HillaryClinton, \#Debates, \#Hillary2016,
\#DonaldTrump and \#Trump2016.

The time range includes the 90 minutes of the presidential debate, as
well as 15 minutes before and 15 minutes after the debate. The resulting
dataset contains 6,498,818 tweets, emitted by 1,451,388 twitter users.
For each user, the Twitter API provides aggregate information such as
the number of followers, the total number (over the lifetime of the
user) of emitted tweets, authored retweets, and favorites. For
individual tweets, the API provides the timestamp and, if it is a
retweet, the original tweet that started the retweet cascade.

\hypertarget{installation-of-r-and-rstudio}{%
\subsection{Installation of R and
RStudio}\label{installation-of-r-and-rstudio}}

This tutorial will be conducted using the R programming language. You
will need to download the \href{https://cran.r-project.org/}{R base
package} and optionally
\href{https://www.rstudio.com/products/rstudio/download/}{RStudio} if
you prefer a graphical user interface. Personally, I would encourage you
to install RStudio as it makes project management much easier. It will
also mean that you can open this tutorial RMarkdown file in RStudio and
run the code directly.

Tip: Create a new \emph{project} in RStudio where you can store your
files for this project.

\hypertarget{quick-overview-of-r}{%
\subsection{Quick overview of R}\label{quick-overview-of-r}}

In this section we will cover some basic concepts in the R language to
get you up and running with it.

\hypertarget{variables}{%
\subsubsection{3.1 Variables}\label{variables}}

Like any programming language, you write code and execute it in the
console. A difference between other languages is the assignment
operator:

\begin{Shaded}
\begin{Highlighting}[]
\NormalTok{myVariable <-}\StringTok{ "Hello, world!"}
\end{Highlighting}
\end{Shaded}

What is that strange notation: \texttt{\textless{}-} ? This is known as
an assignment operator. In some ways \texttt{\textless{}-} is quite
similar to \texttt{=}. However, there are important differences, which
is why we use the \texttt{\textless{}-} notation rather than \texttt{=}.

\hypertarget{data-structures-and-data-types}{%
\subsubsection{Data structures and data
types}\label{data-structures-and-data-types}}

\hypertarget{vectors}{%
\paragraph{Vectors}\label{vectors}}

Probably the most important/common data structure in R are vectors. In
fact, we have been already working with vectors. If you have been
working through the previous examples, try typing in the following:

\begin{Shaded}
\begin{Highlighting}[]
\KeywordTok{is.vector}\NormalTok{(myVariable)}
\end{Highlighting}
\end{Shaded}

In R, a vector is a set of elements that are most commonly character,
logical, integer or numeric.

Here is a simple numeric vector:

\begin{Shaded}
\begin{Highlighting}[]
\NormalTok{x <-}\StringTok{ }\DecValTok{5}
\NormalTok{x}
\end{Highlighting}
\end{Shaded}

Here is a character vector:

\begin{Shaded}
\begin{Highlighting}[]
\NormalTok{myName <-}\StringTok{ "John"}
\NormalTok{myName}
\end{Highlighting}
\end{Shaded}

We can look at attributes of vectors, e.g.~finding out how many
characters \texttt{myName} is. In the following code, we are using the
\texttt{nchar} function and providing as input our \texttt{myName}
variable.

\begin{Shaded}
\begin{Highlighting}[]
\KeywordTok{nchar}\NormalTok{(myName)}
\end{Highlighting}
\end{Shaded}

We can also assign `logical' or `boolean' vectors, which are either TRUE
or FALSE:

\begin{Shaded}
\begin{Highlighting}[]
\NormalTok{skyIsBlue <-}\StringTok{ }\OtherTok{TRUE}
\NormalTok{skyIsBlue}
\end{Highlighting}
\end{Shaded}

Then we can ask R to tell us whether \texttt{skyIsBlue} is TRUE or not,
by feeding it into the \texttt{isTRUE} function:

\begin{Shaded}
\begin{Highlighting}[]
\KeywordTok{isTRUE}\NormalTok{(skyIsBlue)}
\end{Highlighting}
\end{Shaded}

We don't have to have only one element in a vector. Often we want to use
multiple elements, e.g.~creating a numeric vector with the numbers 1
through 5.

\begin{Shaded}
\begin{Highlighting}[]
\NormalTok{countToFive <-}\StringTok{ }\KeywordTok{c}\NormalTok{(}\DecValTok{1}\NormalTok{,}\DecValTok{2}\NormalTok{,}\DecValTok{3}\NormalTok{,}\DecValTok{4}\NormalTok{,}\DecValTok{5}\NormalTok{)}
\NormalTok{countToFive}
\end{Highlighting}
\end{Shaded}

We can also use shorthand notation to do the same thing:

\begin{Shaded}
\begin{Highlighting}[]
\NormalTok{countToTen <-}\StringTok{ }\DecValTok{1}\OperatorTok{:}\DecValTok{10}
\NormalTok{countToTen}
\end{Highlighting}
\end{Shaded}

We can find out how many elements are in the numeric vector:

\begin{Shaded}
\begin{Highlighting}[]
\KeywordTok{length}\NormalTok{(countToTen)}
\end{Highlighting}
\end{Shaded}

We can access the fifth element of \texttt{countToTen} using the square
brackets notation. This indexes \texttt{countToTen} and looks for the
fifth element, and returns the value to us.

\begin{Shaded}
\begin{Highlighting}[]
\NormalTok{countToTen[}\DecValTok{5}\NormalTok{]}
\end{Highlighting}
\end{Shaded}

We can also access the first 3 elements:

\begin{Shaded}
\begin{Highlighting}[]
\NormalTok{countToTen[}\DecValTok{1}\OperatorTok{:}\DecValTok{3}\NormalTok{]}
\end{Highlighting}
\end{Shaded}

We can find out information about our \texttt{countToTen} vector. The
\texttt{typeof} function gives us basic information about the type of
object. The \texttt{str} function is useful for finding out more detail
what an object is. This tells us that it is of type `int' (integer).

\begin{Shaded}
\begin{Highlighting}[]
\KeywordTok{typeof}\NormalTok{(countToTen)}
\KeywordTok{str}\NormalTok{(countToTen)}
\end{Highlighting}
\end{Shaded}

We can also examine the \texttt{myName} character vector that we created
earlier. Note the difference, namely that is of type `chr' (character).

\begin{Shaded}
\begin{Highlighting}[]
\KeywordTok{str}\NormalTok{(myName)}
\end{Highlighting}
\end{Shaded}

\hypertarget{matrices}{%
\paragraph{Matrices}\label{matrices}}

Matrices are special vectors in R. They are `atomic', so they can only
contain data of one type (e.g.~you can't have a column with integer data
and another column with character data).

Matrices are filled column-wise, for example:

\begin{Shaded}
\begin{Highlighting}[]
\NormalTok{myMatrix <-}\StringTok{ }\KeywordTok{matrix}\NormalTok{(}\DecValTok{1}\OperatorTok{:}\DecValTok{6}\NormalTok{, }\DataTypeTok{nrow =} \DecValTok{2}\NormalTok{, }\DataTypeTok{ncol =} \DecValTok{3}\NormalTok{)}
\NormalTok{myMatrix}
\end{Highlighting}
\end{Shaded}

You can make a matrix out of two vector objects, for example:

\begin{Shaded}
\begin{Highlighting}[]
\NormalTok{vector1 <-}\StringTok{ }\DecValTok{1}\OperatorTok{:}\DecValTok{5}
\NormalTok{vector2 <-}\StringTok{ }\DecValTok{6}\OperatorTok{:}\DecValTok{10}
\NormalTok{myMatrix2 <-}\StringTok{ }\KeywordTok{cbind}\NormalTok{(vector1,vector2)}
\NormalTok{myMatrix2}
\end{Highlighting}
\end{Shaded}

There are some new things here. We defined two numeric vectors
\texttt{vector1} and \texttt{vector2}. Then we used the \texttt{cbind}
function to `stick' these together column-wise (i.e.~vertically next to
each other). The result is stored in a new variable called
\texttt{myMatrix2}, which is a matrix, because it is a multi-dimensional
numeric vector. The matrix also has names for the columns, which it gets
automatically from the names of the variables \texttt{vector1} and
\texttt{vector2}.

We can access the column names of myMatrix2:

\begin{Shaded}
\begin{Highlighting}[]
\KeywordTok{colnames}\NormalTok{(myMatrix2)}
\end{Highlighting}
\end{Shaded}

We access the element in the first row and second column using the
square bracket notation for accessing particular elements (i.e.~fishing
out the values of particular cells of the matrix). Within the square
brackets we have the row number on the left hand side of the comma, and
the column number of the right hand side of the comma. For example, if
we want to access the value of the cell in the first row and second
column we would use:

\begin{Shaded}
\begin{Highlighting}[]
\NormalTok{myMatrix2[}\DecValTok{1}\NormalTok{,}\DecValTok{2}\NormalTok{]}
\end{Highlighting}
\end{Shaded}

We can access all of the second column of our matrix by simply not
providing a row number (i.e.~leaving it empty on the left hand side of
the comma):

\begin{Shaded}
\begin{Highlighting}[]
\NormalTok{myMatrix2[,}\DecValTok{2}\NormalTok{]}
\end{Highlighting}
\end{Shaded}

\hypertarget{lists}{%
\paragraph{Lists}\label{lists}}

Lists in R are fairly similar to other languages. They can contain
elements of different types (unlike vectors and matrices).

\begin{Shaded}
\begin{Highlighting}[]
\NormalTok{myList <-}\StringTok{ }\KeywordTok{list}\NormalTok{(}\StringTok{"Hello"}\NormalTok{, }\DecValTok{1}\NormalTok{, }\OtherTok{TRUE}\NormalTok{, }\StringTok{"Goodbye"}\NormalTok{)}
\NormalTok{myList}
\end{Highlighting}
\end{Shaded}

We access elements in lists slightly differently to vectors, but still
use the square brackets notation.

\begin{Shaded}
\begin{Highlighting}[]
\NormalTok{myList[}\DecValTok{1}\NormalTok{]}
\end{Highlighting}
\end{Shaded}

We are accessing the first `slice' of \texttt{myList}, which is a list
which itself also contains one element, namely the character vector
``Hello''. The list contains nested elements.

We can get more specific and access the ``Hello'' element, using double
square brackets notation. The next line of code accesses the first slice
of \texttt{myList}, accessed using the subset \texttt{{[}1{]}}, and then
the first element within that slice, accessed using the subset
\texttt{{[}{[}1{]}{]}}. We are subsetting \texttt{myList} twice, once
with \texttt{{[}1{]}} and secondly with \texttt{{[}{[}1{]}{]}}.

\begin{Shaded}
\begin{Highlighting}[]
\NormalTok{myList[[}\DecValTok{1}\NormalTok{]][}\DecValTok{1}\NormalTok{]}
\end{Highlighting}
\end{Shaded}

Similarly, we access the first element of the \emph{second} slice of
\texttt{myList} like this:

\begin{Shaded}
\begin{Highlighting}[]
\NormalTok{myList[[}\DecValTok{2}\NormalTok{]][}\DecValTok{1}\NormalTok{]}
\end{Highlighting}
\end{Shaded}

\hypertarget{factors}{%
\paragraph{Factors}\label{factors}}

In R, factors are special vectors that represent categorical data.

Here we create a factor \texttt{myFactor} that provides data on 5
students and whether each one passed or failed their assignments. We can
see that there are two `levels' to this factor, namely ``pass'' or
``fail''. (This is a very badly performing sample of students!)

\begin{Shaded}
\begin{Highlighting}[]
\NormalTok{myFactor <-}\StringTok{ }\KeywordTok{factor}\NormalTok{(}\KeywordTok{c}\NormalTok{(}\StringTok{"pass"}\NormalTok{, }\StringTok{"fail"}\NormalTok{, }\StringTok{"fail"}\NormalTok{, }\StringTok{"pass"}\NormalTok{, }\StringTok{"fail"}\NormalTok{))}
\NormalTok{myFactor}
\end{Highlighting}
\end{Shaded}

We can subset the factor to find which students are ``fail''. This
returns the indexes of the elements in \texttt{myFactor} that equal
``fail''. Notice that the equals sign is \texttt{==} (not \texttt{=}).
This double equals sign is used to test for equality. Here we are asking
the question: which elements of \texttt{myFactor} are equal to ``fail''?
If there are any elements that match, then it returns the indexes of
these elements. Sure as eggs, they match up with what we expect (the
2nd, 3rd, and 5th elements are fail).

\begin{Shaded}
\begin{Highlighting}[]
\KeywordTok{which}\NormalTok{(myFactor}\OperatorTok{==}\StringTok{"fail"}\NormalTok{)}
\end{Highlighting}
\end{Shaded}

\hypertarget{data-frames}{%
\paragraph{3.3.5 Data frames}\label{data-frames}}

Data frames are very important in R, and we will use them a lot.

Data frames are similar to matrices, in that they are often
two-dimensional with rows and columns. Roughly speaking, the biggest
difference between data frames and matrices is that data frames can
contain columns \emph{with different types of data}.

In this way, each column in the data frame can have a different data
type, for example:

\begin{Shaded}
\begin{Highlighting}[]
\NormalTok{df <-}\StringTok{ }\KeywordTok{data.frame}\NormalTok{(}\DataTypeTok{names =} \KeywordTok{c}\NormalTok{(}\StringTok{"John"}\NormalTok{,}\StringTok{"Jane"}\NormalTok{,}\StringTok{"Sally"}\NormalTok{), }
  \DataTypeTok{testScores=}\KeywordTok{c}\NormalTok{(}\DecValTok{99}\NormalTok{,}\DecValTok{84}\NormalTok{,}\DecValTok{30}\NormalTok{), }\DataTypeTok{failingGrade=}\KeywordTok{c}\NormalTok{(}\OtherTok{FALSE}\NormalTok{, }\OtherTok{FALSE}\NormalTok{, }\OtherTok{TRUE}\NormalTok{))}
\NormalTok{df}
\end{Highlighting}
\end{Shaded}

We can access the data in the third column in two ways. First, we can
use the brackets notation:

\begin{Shaded}
\begin{Highlighting}[]
\NormalTok{df[,}\DecValTok{3}\NormalTok{]}
\end{Highlighting}
\end{Shaded}

We can also use the dollar sign notation to do the same thing. We know
that the third column of \texttt{df} has name, `failingGrade', so we can
subset the data frame by name using the dollar sign notation:

\begin{Shaded}
\begin{Highlighting}[]
\NormalTok{df}\OperatorTok{$}\NormalTok{failingGrade}
\end{Highlighting}
\end{Shaded}

We can find the number of rows in data frame by calling the
\texttt{nrow} function and supplying \texttt{df} as input to it:

\begin{Shaded}
\begin{Highlighting}[]
\KeywordTok{nrow}\NormalTok{(df)}
\end{Highlighting}
\end{Shaded}

We can also find out the structure of each column of \texttt{df} using
the \texttt{str} function:

\begin{Shaded}
\begin{Highlighting}[]
\KeywordTok{str}\NormalTok{(df)}
\end{Highlighting}
\end{Shaded}

Finally, we can view the data frame quite nicely using the \texttt{View}
function. This is especially useful for working with large data frames
that we will be working with in the remainder of this tutorial.

\begin{Shaded}
\begin{Highlighting}[]
\KeywordTok{View}\NormalTok{(df)}
\end{Highlighting}
\end{Shaded}

\hypertarget{loading-and-wrangling-the-debatenight-dataset}{%
\subsection{Loading and wrangling the \#DEBATENIGHT
dataset}\label{loading-and-wrangling-the-debatenight-dataset}}

Now that you are familiar with R syntax we will import the dataset and
start analysing it.

Download the dataset and save it to your local working directory in R.
The dataset can be downloaded from:
\url{http://128.199.169.209/sample_users_100k.csv.bz2} .

\begin{Shaded}
\begin{Highlighting}[]
\NormalTok{data_df <-}\StringTok{ }\KeywordTok{read.csv}\NormalTok{(}\StringTok{"sample_users_100k.csv.bz2"}\NormalTok{,}\DataTypeTok{sep=}\StringTok{"}\CharTok{\textbackslash{}t}\StringTok{"}\NormalTok{,}\DataTypeTok{stringsAsFactors =}\NormalTok{ F)}
\end{Highlighting}
\end{Shaded}

\begin{verbatim}
## Warning in file(file, "rt"): cannot open file 'sample_users_100k.csv.bz2':
## No such file or directory
\end{verbatim}

\begin{verbatim}
## Error in file(file, "rt"): cannot open the connection
\end{verbatim}

We also want to install an external R package called \texttt{Rmisc}. The
Rmisc library contains many functions useful for data analysis and
utility operations.

\begin{Shaded}
\begin{Highlighting}[]
\CommentTok{# install.packages("Rmisc")}
\KeywordTok{library}\NormalTok{(Rmisc)}
\end{Highlighting}
\end{Shaded}

\begin{verbatim}
## Error in library(Rmisc): there is no package called 'Rmisc'
\end{verbatim}

We will find out some information about our data:

\begin{Shaded}
\begin{Highlighting}[]
\KeywordTok{str}\NormalTok{(data_df)}
\end{Highlighting}
\end{Shaded}

\begin{verbatim}
## Error in str(data_df): object 'data_df' not found
\end{verbatim}

We can view the dataframe with the following command. It is useful to
have this open so you can refer to it.

\begin{Shaded}
\begin{Highlighting}[]
\KeywordTok{View}\NormalTok{(data_df)}
\end{Highlighting}
\end{Shaded}

\begin{verbatim}
## Error in as.data.frame(x): object 'data_df' not found
\end{verbatim}

A super useful function in R is \texttt{summary()}, which takes care of
many descriptive statistics of interest for each variable:

\begin{Shaded}
\begin{Highlighting}[]
\KeywordTok{summary}\NormalTok{(data_df)}
\end{Highlighting}
\end{Shaded}

\begin{verbatim}
## Error in summary(data_df): object 'data_df' not found
\end{verbatim}

The \texttt{summarySE} function in the \texttt{Rmisc} package outputs
the number of observations, mean, standard deviation, standard error of
the mean, and confidence interval for grouped data. The summarySE
function allows you to summarize over the combination of multiple
independent variables by listing them as a vector,
e.g.~c(``friendsCount'', ``followersCount'').

First we will coerce the \texttt{verified} variable to a factor and use
the \texttt{table} function to look at the distribution.

\begin{Shaded}
\begin{Highlighting}[]
\NormalTok{data_df}\OperatorTok{$}\NormalTok{verified <-}\StringTok{ }\KeywordTok{as.factor}\NormalTok{(data_df}\OperatorTok{$}\NormalTok{verified)}
\end{Highlighting}
\end{Shaded}

\begin{verbatim}
## Error in is.factor(x): object 'data_df' not found
\end{verbatim}

\begin{Shaded}
\begin{Highlighting}[]
\KeywordTok{table}\NormalTok{(data_df}\OperatorTok{$}\NormalTok{verified)}
\end{Highlighting}
\end{Shaded}

\begin{verbatim}
## Error in table(data_df$verified): object 'data_df' not found
\end{verbatim}

Then we call the \texttt{summarySE} function to look at how the number
of followers of users (followersCount) varies depending on whether the
user is `verified' or not (has established their true identity with
Twitter). Obviously, celebrities and public figures will be more likely
to verify themselves, so we can see that the followers numbers are much
higher for verified users.

At the same time, we see that standard error (SE) and confidence
intervals (CI) are also higher for the sample of verified users.

Remember when it comes to standard error and confidence interval,
smaller is better: smaller values mean the more representative the
sample will be of the overall population.

\begin{Shaded}
\begin{Highlighting}[]
\KeywordTok{summarySE}\NormalTok{(}\DataTypeTok{data=}\NormalTok{data_df,}
          \StringTok{"followersCount"}\NormalTok{,}
          \DataTypeTok{groupvars=}\StringTok{"verified"}\NormalTok{,}
          \DataTypeTok{conf.interval =} \FloatTok{0.95}\NormalTok{, }\DataTypeTok{na.rm =}\NormalTok{ T)}
\end{Highlighting}
\end{Shaded}

\begin{verbatim}
## Error in summarySE(data = data_df, "followersCount", groupvars = "verified", : could not find function "summarySE"
\end{verbatim}

One of the variables we are particularly interested in is the
\texttt{botscore}. Obviously, there is an issue with the summary
statistics for this variable, because it should be numeric and the range
should be 0 to 1.

\begin{Shaded}
\begin{Highlighting}[]
\KeywordTok{summary}\NormalTok{(data_df}\OperatorTok{$}\NormalTok{botscore)}
\end{Highlighting}
\end{Shaded}

\begin{verbatim}
## Error in summary(data_df$botscore): object 'data_df' not found
\end{verbatim}

As you can see, running the summary doesn't tell us much! We can see
that the structure of this variable is of type \texttt{character}, and
there are some clearly non-numeric values.

\begin{Shaded}
\begin{Highlighting}[]
\KeywordTok{str}\NormalTok{(data_df}\OperatorTok{$}\NormalTok{botscore)}
\end{Highlighting}
\end{Shaded}

\begin{verbatim}
## Error in str(data_df$botscore): object 'data_df' not found
\end{verbatim}

So, let's wrangle this variable and fix it up. First, we will coerce it
from character to numeric. Note the warning from R telling us that it
had to introduce \texttt{NA} values. These \texttt{NA} values denote
missing data and were previously the elements that were not numeric
(e.g. `deleted', `suspended').

\begin{Shaded}
\begin{Highlighting}[]
\NormalTok{data_df}\OperatorTok{$}\NormalTok{botscore <-}\StringTok{ }\KeywordTok{as.numeric}\NormalTok{(data_df}\OperatorTok{$}\NormalTok{botscore)}
\end{Highlighting}
\end{Shaded}

\begin{verbatim}
## Error in eval(expr, envir, enclos): object 'data_df' not found
\end{verbatim}

When we re-run summary, we also notice that range has a problem with
some erroneous data creeping (probably due to the coercion), with values
of minus infinity (!) which R understands as \texttt{-Inf}. Let's fix
that too.

\begin{Shaded}
\begin{Highlighting}[]
\NormalTok{toDel <-}\StringTok{ }\KeywordTok{which}\NormalTok{(data_df}\OperatorTok{$}\NormalTok{botscore }\OperatorTok{<}\StringTok{ }\DecValTok{0}\NormalTok{)}
\end{Highlighting}
\end{Shaded}

\begin{verbatim}
## Error in which(data_df$botscore < 0): object 'data_df' not found
\end{verbatim}

\begin{Shaded}
\begin{Highlighting}[]
\NormalTok{data_df <-}\StringTok{ }\NormalTok{data_df[}\OperatorTok{-}\NormalTok{toDel,]}
\end{Highlighting}
\end{Shaded}

\begin{verbatim}
## Error in eval(expr, envir, enclos): object 'data_df' not found
\end{verbatim}

Now we have our correct bot scores:

\begin{Shaded}
\begin{Highlighting}[]
\KeywordTok{summary}\NormalTok{(data_df}\OperatorTok{$}\NormalTok{botscore)}
\end{Highlighting}
\end{Shaded}

\begin{verbatim}
## Error in summary(data_df$botscore): object 'data_df' not found
\end{verbatim}

\begin{Shaded}
\begin{Highlighting}[]
\KeywordTok{plot}\NormalTok{(}\KeywordTok{sort}\NormalTok{(data_df}\OperatorTok{$}\NormalTok{botscore),}\DataTypeTok{main=}\StringTok{"Distribution of bot scores by users"}\NormalTok{,}
        \DataTypeTok{ylab =} \StringTok{"Bot score"}\NormalTok{, }\DataTypeTok{xlab =} \StringTok{"# User"}\NormalTok{)}
\end{Highlighting}
\end{Shaded}

\begin{verbatim}
## Error in sort(data_df$botscore): object 'data_df' not found
\end{verbatim}

A more instructive view of the bot scores can be gained through box
plots:

\begin{Shaded}
\begin{Highlighting}[]
\KeywordTok{boxplot}\NormalTok{(data_df}\OperatorTok{$}\NormalTok{botscore,}\DataTypeTok{main=}\StringTok{"Boxplot of bot score data"}\NormalTok{,}\DataTypeTok{ylab=}\StringTok{"Bot score"}\NormalTok{)}
\end{Highlighting}
\end{Shaded}

\begin{verbatim}
## Error in boxplot(data_df$botscore, main = "Boxplot of bot score data", : object 'data_df' not found
\end{verbatim}

You can see there are many scores that are determined as outliers. In
order to be an outlier, the data value must be:

\begin{itemize}
\tightlist
\item
  larger than Q3 by at least 1.5 times the interquartile range (IQR), or
\item
  smaller than Q1 by at least 1.5 times the IQR.
\end{itemize}

Remember that the IQR = Q3 minus Q1.

Another way to view the bot score data is using a histogram:

\begin{Shaded}
\begin{Highlighting}[]
\KeywordTok{hist}\NormalTok{(data_df}\OperatorTok{$}\NormalTok{botscore,}\DataTypeTok{main=}\StringTok{"Bot score histogram"}\NormalTok{)}
\end{Highlighting}
\end{Shaded}

\begin{verbatim}
## Error in hist(data_df$botscore, main = "Bot score histogram"): object 'data_df' not found
\end{verbatim}

A simple tweak to the function call lets you estimate and visualise a
density plot:

\begin{Shaded}
\begin{Highlighting}[]
\NormalTok{densityVal <-}\StringTok{ }\KeywordTok{density}\NormalTok{(data_df}\OperatorTok{$}\NormalTok{botscore,}\DataTypeTok{na.rm =}\NormalTok{ T)}
\end{Highlighting}
\end{Shaded}

\begin{verbatim}
## Error in density(data_df$botscore, na.rm = T): object 'data_df' not found
\end{verbatim}

\begin{Shaded}
\begin{Highlighting}[]
\KeywordTok{plot}\NormalTok{(densityVal,}\DataTypeTok{type=}\StringTok{"n"}\NormalTok{, }\DataTypeTok{main=}\StringTok{"Density plot of bot scores"}\NormalTok{)}
\end{Highlighting}
\end{Shaded}

\begin{verbatim}
## Error in plot(densityVal, type = "n", main = "Density plot of bot scores"): object 'densityVal' not found
\end{verbatim}

\begin{Shaded}
\begin{Highlighting}[]
\KeywordTok{polygon}\NormalTok{(densityVal, }\DataTypeTok{col=}\StringTok{"red"}\NormalTok{, }\DataTypeTok{border=}\StringTok{"gray"}\NormalTok{)}
\end{Highlighting}
\end{Shaded}

\begin{verbatim}
## Error in xy.coords(x, y, setLab = FALSE): object 'densityVal' not found
\end{verbatim}

\hypertarget{inferential-statistics}{%
\subsection{Inferential statistics}\label{inferential-statistics}}

In the lecture we learned about inferential statistics. When trying to
determine whether two variables are related (e.g., class attendance and
assignment grades), statistical testing of hypotheses is the tool most
frequently used. To test this, two types of hypotheses are considered,
the research or alternative hypothesis and the null hypothesis.

In this section we will do some statistical tests to look at
relationships between variables in the dataset.

Although we could use a variety of tests, in this lab we will focus on
using the t-test (as described in the lecture).

\hypertarget{t-test-example}{%
\subsubsection{T-test example}\label{t-test-example}}

Suppose that we wish to find out whether there is a statistically
significant difference between the mean of the ratio of friends
(\texttt{friendsCount}) to followers (\texttt{followersCount}) of
\emph{humans} VS \emph{bots}. Let's call this ratio \(psi\).

Our intuition is that bots will follow other users but not attract as
many followers back, whereas humans will tend to have a greater
reciprocity of social ties. The \(psi\) metric is a crude way to measure
this.

Our research hypothesis might something like:

\textbf{Ha: Bots will have a higher friends/followers ratio than
humans.}

The null hypothesis therefore:

\textbf{Ho: There is no significant difference in \(psi\) between humans
and bots.}

First we calculate the ratio \(psi\) and store it in the dataframe. To
avoid a situation of dividing by zero, we make a small adjustment to the
formula.

\begin{Shaded}
\begin{Highlighting}[]
\NormalTok{data_df}\OperatorTok{$}\NormalTok{friendsFollowersRatio <-}\StringTok{ }\NormalTok{data_df}\OperatorTok{$}\NormalTok{friendsCount }\OperatorTok{/}\StringTok{ }\NormalTok{(data_df}\OperatorTok{$}\NormalTok{followersCount }\OperatorTok{+}\StringTok{ }\FloatTok{0.01}\NormalTok{)}
\end{Highlighting}
\end{Shaded}

\begin{verbatim}
## Error in eval(expr, envir, enclos): object 'data_df' not found
\end{verbatim}

Let's now examine the means of the bot group versus the human group. We
will use a very simple threshold. If the score is less than or equal to
0.5, then \emph{human}; otherwise \emph{bot}.

\begin{Shaded}
\begin{Highlighting}[]
\CommentTok{# Human mean friend/follower ratio:}
\KeywordTok{mean}\NormalTok{(data_df}\OperatorTok{$}\NormalTok{friendsFollowersRatio[data_df}\OperatorTok{$}\NormalTok{botscore }\OperatorTok{<=}\StringTok{ }\FloatTok{0.5}\NormalTok{],}\DataTypeTok{na.rm =}\NormalTok{ T)}
\end{Highlighting}
\end{Shaded}

\begin{verbatim}
## Error in mean(data_df$friendsFollowersRatio[data_df$botscore <= 0.5], : object 'data_df' not found
\end{verbatim}

\begin{Shaded}
\begin{Highlighting}[]
\CommentTok{# Bot mean friend/follower ratio:}
\KeywordTok{mean}\NormalTok{(data_df}\OperatorTok{$}\NormalTok{friendsFollowersRatio[data_df}\OperatorTok{$}\NormalTok{botscore }\OperatorTok{>}\StringTok{ }\FloatTok{0.5}\NormalTok{],}\DataTypeTok{na.rm =}\NormalTok{ T)}
\end{Highlighting}
\end{Shaded}

\begin{verbatim}
## Error in mean(data_df$friendsFollowersRatio[data_df$botscore > 0.5], na.rm = T): object 'data_df' not found
\end{verbatim}

This suggests that bots have a much more positively skewed \(psi\)
ratio. But how can we be sure it's not due to chance? We use the t-test.

We want to specify a one-sided t-test because we want to establish not
simply whether there is \emph{any} significant relationship between
\(psi\) and bot score but also the direction of the relationship. We are
hypothesising that when the friends/followers ratio \(psi\) is high, the
bot score is also high. The parameter of the \texttt{t.test()} function
for specifying this one-sided test is the \texttt{alternative}
parameter, and we pass the ``greater'' argument to it.

\begin{Shaded}
\begin{Highlighting}[]
\KeywordTok{t.test}\NormalTok{(data_df}\OperatorTok{$}\NormalTok{friendsFollowersRatio[data_df}\OperatorTok{$}\NormalTok{botscore }\OperatorTok{<=}\StringTok{ }\FloatTok{0.5}\NormalTok{],data_df}\OperatorTok{$}\NormalTok{friendsFollowersRatio[data_df}\OperatorTok{$}\NormalTok{botscore }\OperatorTok{>}\StringTok{ }\FloatTok{0.5}\NormalTok{],}\DataTypeTok{alternative =} \StringTok{"less"}\NormalTok{)}
\end{Highlighting}
\end{Shaded}

\begin{verbatim}
## Error in t.test(data_df$friendsFollowersRatio[data_df$botscore <= 0.5], : object 'data_df' not found
\end{verbatim}

From these results we can see that \texttt{t} (the t-ratio) is quite
large and \texttt{p} (the p-value) is very small and much lower than
\textless{} 0.001. \textbf{This is a significant result!}. Therefore, we
can reject the null hypothesis and accept the research or alternative
hypothesis, thus concluding that bots are more likely to have a higher
\(psi\).

Additionally: the null hypothesis is that the mean of \(psi\) for bots
is 6.492005. The alternate hypothesis is that the data come from a
distribution with mean greater than 6.492005.

The test rejected the null hypothesis, and the 95\% confidence interval
is that the mean is greater than 6.492005, which equivalent to saying it
is in the interval {[}8.736995,Inf{]}.

\hypertarget{data-exploration-and-analysis}{%
\subsection{Data exploration and
analysis}\label{data-exploration-and-analysis}}

For the remainder of this lab session I encourage you to play around
with the dataset, and to try out different things with it. You may wish
to:

\begin{itemize}
\tightlist
\item
  wrangle or clean variables in order to use them in analysis
\item
  calculate some distributions and do some plots (e.g.~histograms and
  boxplots)
\item
  do some statistical tests between variables and see whether you can
  come up with significant p-values!
\end{itemize}

Have a look at the following webpage for a list of statistical tests you
can experiment with in R:
\url{http://r-statistics.co/Statistical-Tests-in-R.html} .

The machine learning course next week will focus on linear regression,
which follows on naturally from what you have learned here. You will
move into the area of \emph{prediction} in the next lecture and lab. You
can get a head start on the assignment by learning about and describing
your data and testing some of the relationships in it!

\hypertarget{conclusion}{%
\subsection{Conclusion}\label{conclusion}}

Thanks for attending today. If you have any questions please contact me
(\href{mailto:Dongwoo.Kim@anu.edu.au}{\nolinkurl{Dongwoo.Kim@anu.edu.au}}).

Dongwoo


\end{document}
